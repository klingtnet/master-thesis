\chapter*{Abstract}\label{abstract}
\addcontentsline{toc}{chapter}{Abstract}

% use text from introduction

\section*{English}

Software Audiosynthesizers have gained in popularity over the past 10 years and it is impossible to imagine professional or home studios without them.
This popularity is mostly justified by the high computing power, which is available everywhere on PCs and mobile devices and makes real-time audio synthesis usable.
The aim of this work is the detailed description of basic synthesizer components and the investigation of suitable algorithms and techniques for their realization.

In the first part, input protocols are considered and the fundamental building blocks of a synthiser are introduced.
Subsequently, different synthesis techniques are explained and the decision for the choice of the subtractive synthesis is explained.

The second part deals with signal processing topics of the individual synthesizer components.
The basics of digital filters are explained and FIR and IIR filters are compared.
Thereafter, the audio quality and efficiency of various waveform synthesis methods is evaluated, i. a. band-limited impulse trains (BLIT) and wavetables.
Implementation details are explained in chapter 7 and the components of the synthesizer application are evaluated.

It was found through the investigation of the synthesizer implementation that the selected techniques and algorithms have high audio quality and low calculation costs, in particular Rust has proven to be a very suitable choice for the development of real-time applications.

\section*{German}

Software Audiosynthesizer haben in den letzten 10 Jahren enorm an Popularität gewonnen und sind in vielen Profi- und Heimstudios nicht mehr wegzudenken.
Diese Popularität ist durch die hohe Rechenleistung begründet, welche auf PCs und mobilen Geräten überall zur Verfügung steht und Echtzeitaudiosynthese nutzbar macht.
Das Ziel dieser Arbeit ist die ausführliche Beschreibung grundlegender Synthesizerkomponenten und die Untersuchung geeigneter Algorithmen und Techniken für deren Realisierung.

Im ersten Teil der Arbeit werden Eingabeprotokolle betrachtet und die fundamentalen Bausteine eines Synthersizers eingeführt.
Anschließend werden verschiedene Synthesetechniken erläutert und die Entscheidung für die Wahl der subtraktiven Synthese begründet.

Der zweite Teil beschäftigt sich mit der Signalverarbeitung innerhalb der einzelnen Synthesizerkomponenten.
Die Grundlagen von digitalen Filtern werden erläutert und FIR mit IIR Filtern verglichen.
Nachfolgend wird die Audioqualität und Effizienz verschiedener Wellenformsynthesemethoden, u.a. bandbeschränkte Impulsfolgen (BLIT) und Wavetables, evaluiert.
Implementierungsdetails werden im 7. Kapitel erläutert und die Komponenten des entwickelten Synthesizers ausgewertet.

Es zeigte sich durch Untersuchung der Synthesizerimplementation, dass die ausgewählten Techniken und Algorithmen eine hohe Audioqualität bei gleichzeitig niedrigen Berechnungskosten haben, insbesondere Rust hat sich als sehr geeignete Wahl für die Entwicklung von Echtzeitanwendungen erwiesen.

\begin{center}
	{\sffamily
		\textbf{\documentAuthor{}}\\
		\textsc{\university{}}\\
		\universityFaculty{}---\universityDepartment{}\\
		\documentTitle{}\\
		\germanDate{}\today{}\\
		\bigskip{}
	}
\end{center}