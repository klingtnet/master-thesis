\chapter*{Abstract}\label{abstract}
\addcontentsline{toc}{chapter}{Abstract}

\section*{German}

Im Rahmen dieser Arbeit wird der Entwurf einesAudio-Synthesizers dokumentiert, welcher über im Audiobereich übliche Protokolle ferngesteuert werden kann und dessen Ausgabe in Echtzeit stattfindet.

Es werden die Eingabeprotokolle MIDI und Open Sound Control (OSC) erläutert und deren Vor- und Nachteile gegenübergestellt.
Nachfolgend werden die Grundbausteine von Synthesizern, darunter Oszillatoren und Aliasingprobleme von bestimmten Wellenformen, die nicht-linearität des menschlichen Gehörs und Hüllkurvengeneratoren beschrieben.
Daraufhin werden übliche Synthesetechniken wie Additive und Subtraktive und FM-Synthese erklärt.

Grundlagen der digital Signalverarbeitung werden eingeführt, digital Filter klassifiziert und IIR als auch FIR Filter Topologien evaluiert.
Anschließend werden verschiedene Oszillatoralgorithmen besprochen und nach deren Aliasing-Anteil und Rechenaufwand klassifiziert und betrachtet.
Danach werden Probleme bei der Implementierung diskutiert und Lösungen vorgeschlagen sowie Implementierungsentscheidungen erläutert.

Schlussendlich werden mögliche Optimierungen vorgeschlagen und der Zustand der Implementierung mit den Vorgaben verglichen und ausgewertet.

\section*{English}

Within the scope of this thesis, the design of an audio synthesizer is documented, which can be remotely controlled over commonly used control protocols and whose output is generated in real time.

The input protocols MIDI and Open Sound Control (OSC) are explained and their advantages and disadvantages are compared.
The basic building blocks of synthesizers, including oscillators and aliasing problems of certain waveforms, the non-linearity of human hearing and envelope generators, are described below.
Subsequently, conventional synthesis techniques such as additive, subtractive and FM synthesis are explained.

Basics of digital signal processing are introduced, digital filters are classified and IIR as well as FIR filter topologies are evaluated.
Subsequently, various oscillator algorithms are discussed and classified and examined according to their aliasing and computational complexity.
Afterwards, implementation problems are discussed and solutions are proposed as well as implementation decisions.

In the end, possible optimizations are proposed and the state of the implementation is evaluated against the specifications.

\sffamily{
	\begin{center}
		\textbf{\documentAuthor{}}\\
		\textsc{\university{}}\\
		\universityFaculty{}---\universityDepartment{}\\
		\documentTitle{}\\
		\germanDate{}\today{}\\
		\bigskip{}
	\end{center}
}
